\documentstyle{article}

\title{Device Configuration API}
\author{Bijan Mottahedeh \\ SunSoft, Inc.}
\parindent 0pt
\textwidth 6.5in
\oddsidemargin 0in

\begin{document}
\maketitle

The device configuration (DVC) API provides services that enable an
application to retrieve prototype and actual device configuration data,
add new devices, unconfigure existing devices, modify attributes of
existing devices, and to store the new device configuration data.  The
DVC API gives the application access to two device lists; one, a device
category list which contains names and titles for all available devices
and the other, the configured device list which contains device nodes
for the existing devices in the system.

\section{Device categories}

The device category list contains the names and titles of all devices
which may be configured in the system.  In the DVC API a distinction is
made between names and titles. For example, the category name {\bf
pointer} has the title {\bf Pointing devices}.  The name-title
translation is performed through a map in the file {\bf
/usr/lib/devconfig/abbreviations}.  The name  title mapping for
windowing devices is displayed in figure 1.

\begin{figure}[h]
\begin{center}
\begin{tabular}{@{}|l|l|@{}} \hline
display & Video display adaptors \\ \hline
keyboard & Keyboards \\ \hline
pointer & Pointing devices \\ \hline
\end{tabular} \\
\caption{Name title mapping for windowing devices}
\end{center}
\end{figure}

A device is addressed in the device category list by two indexes; one,
the category index and the other the device index in that
category.  For example, in figure 2, the Logitech bus mouse is
addressed by the index pair \(2,0\) where 2 is the index for the {\bf
Pointing devices} category in the device category list and 0 is the
index of the Logitech bus mouse device in the  {\bf Mouse devices} list. \\

\begin{figure}
\begin{center}
\begin{tabular}{@{}|l|@{}} \hline
\multicolumn{1}{|c|}{Device category} \\ \hline
Keyboards \\
Video display adaptors \\
{\bf Pointing devices} \\ \hline
\end{tabular}
\raisebox{-1.50cm}{\begin{tabular}{@{}|l|@{}} \hline
\multicolumn{1}{|c|}{Mouse devices} \\ \hline
{\bf Logitech bus mouse} \\
Logitech MouseMan serial mouse \\
Microsoft serial mouse \\
Mouse Systems Corp. serial mouse \\ \hline
\end{tabular}} \\
\caption{Device category list}
\end{center}
\end{figure}

{\small \tt \noindent{\begin{tabular}{@{}ll@{}}
void & fetch\_cat\_info(void); \\
void & set\_cat\_idx(int idx); \\
char* & next\_cat\_title(void); \\
char* & next\_cat\_dev\_title(void);
\end{tabular}}} \\

Fetch\_cat\_info() retrieves all device configuration data.  The DVC
API maintains a {\em current category index} which the application sets
in order to address a device in a specific category.
Set\_cat\_idx() sets the current categoy index.  The application builds
its own device category list.  Next\_cat\_title() returns the next
category title in the device category list or NULL if at the end of the
list.  Next\_cat\_dev\_title() returns the next device title in the
current category or NULL if at the end of the list.  For example, the
application may build its category list as follows:

{\small \begin{verbatim}
        for (i = 0; cat_title = next_cat_title(); i++) {
                set_cat_idx(i);
                /* Create a new cateogry at index i */

                for (j = 0; dev_title = next_cat_dev_title(); j++)
                    /* Add the device titles in new category */

        }
\end{verbatim}}

\section{Configured devices}

The configured device list contains nodes for all the configured
devices in the system.  A device in the configured device list is
addressed by its index in the list. In figure 3, the index of
the mouse device is 4. \\

\begin{figure}
\begin{center}
\begin{tabular}{@{}|l|@{}} \hline
\multicolumn{1}{|c|}{Configured devices} \\ \hline
AT COM ports\\
SMC EtherCard PLUS \\
DPT SmartCache Plus, fourth slot/unit \\
ATI 8514/A \\
{\bf Logitech MouseMan serial mouse} \\
AT 101-key keyboard - US English \\ \hline
\end{tabular} \\
\caption{Configured device list}
\end{center}
\end{figure}

{\small \tt \noindent{\mbox{typedef struct device\_info \{}} \\
\begin{tabular}{@{\hspace{1.2cm}}ll@{}}
struct device\_info* & next; \\
char* & name; \\
int & unit; \\
char* & title; \\
attr\_list\_t* & dev\_alist; \\
attr\_list\_t* & typ\_alist; \\
int & modified; \\
void* & ui;
\end{tabular} \\
\mbox{\} device\_info\_t;}} \\

{\small \tt \noindent{\begin{tabular}{@{}ll@{}}
char* & next\_dev\_title(void); \\
device\_info\_t* & make\_dev\_node(int idx); \\
void & free\_dev\_node(device\_info\_t* dp); \\
void & add\_dev\_node(device\_info\_t* dp); \\
void & remove\_dev\_node(int idx); \\
device\_info\_t* & get\_dev\_node(idx); \\
char* & get\_dev\_cat(device\_info\_t* dp);
\end{tabular}}} \\

The application builds its own list of configured devices.
Next\_dev\_title() returns the title of the next device in the
configured device list. For example, the application may build its
configured device list as follows:

{\small \begin{verbatim}
        while (dev_title = next_dev_title())
                /* Add the device title to the list */
\end{verbatim}}

Make\_dev\_node() builds a node for a specified device in the device
catgory list.  Free\_dev\_node() frees a device node.  Add\_dev\_node()
adds a device node to the configured device list.  The application must
not use free\_dev\_node() to free a node in the configured list.  The
configured device list is maintained in an LRU order so the new device
is at index 0; Remove\_dev\_node(idx) removes a device from the
configured device list.  Get\_dev\_node() returns the device node for a
device in the configured device list or NULL if at  the end of the
list.  Get\_dev\_cat() returns the category for a device.  For example,
the application can check whether a device of category {\bf pointer} is
currently configured as follows:

{\small \begin{verbatim}
        for (i = 0; dp = get_dev_node(i); i++)
                dev_cat = get_dev_cat(dp)
                if (!strcmp(dev_cat, "pointer")
                        found_pointer++;
        }
\end{verbatim}}

The Logitech bus mouse in figure 2 is added to the configured device
list as follows:

{\small \begin{verbatim}
        set_cat_idx(2);         /* device category list index */
        dp = make_dev_node(0);  /* device index in the category */

        /* Allow the user to confirm configuration values */

        add_dev_node(dp);        /* add to the configured device list */
\end{verbatim}}

A device node contains two attribute lists, one for the device prototype
data and one for the configured device data.  The attribute list format is
described in the next section. For an existing device, the
configured device data is retrieved from the system. For a new device
the configured device data represents the default choices for the device
attributes.  The application then allows the user to modify the new
device's attributes if necessary.  The {\bf ui} field in the device
node is available for the application and is not used by the DVC API.
For example, the application may allow the user to set proper attributes
for the new Logitech bus mouse as follows:

{\small \begin{verbatim}
        new_device = get_dev_node(0);   /* configured device list index */
        /* Display device attributes to the user */
\end{verbatim}}

\section{Device attributes}

{\small \tt \noindent{\mbox{typedef enum \{}} \\
\begin{tabular}{@{\hspace{1.2cm}}ll@{}}
VAL\_NUMERIC, \\
VAL\_STRING
\end{tabular} \\
\mbox{\} val\_t;}} \\

{\small \tt \noindent{\mbox{typedef union val\_store \{}} \\
\begin{tabular}{@{\hspace{1.2cm}}ll@{}}
char* & string; \\
int &integer;
\end{tabular} \\
\mbox{\} val\_store\_t;}} \\

{\small \tt \noindent{\mbox{typedef struct val\_list \{}} \\
\begin{tabular}{@{\hspace{1.2cm}}ll@{}}
struct val\_list* & next; \\
val\_t & val\_type; \\
val\_store\_t & val;
\end{tabular} \\
\mbox{\} val\_list\_t;}} \\

{\small \tt \noindent{\mbox{typedef struct attr\_list \{}} \\
\begin{tabular}{@{\hspace{1.2cm}}ll@{}}
struct attr\_list* & next; \\
char* & name; \\
val\_list\_t* & vlist;
\end{tabular} \\
\mbox{\} attr\_list\_t;}} \\

{\small \begin{verbatim}
#define NAME_ATTR       "name"
#define WNAME_ATTR      "__owname__"
#define TITLE_ATTR      "__title__"
#define CAT_ATTR        "__category__"
#define INST_ATTR       "__instance__"
#define AUTO_ATTR       "__auto__"
#define REAL_ATTR       "__real__"
#define INSTANCE_ATTR   "__instance__"
#define INTERNAL_ATTR   "^__.*__$"

#define NUMERIC_STRING  "numeric,"
#define STRING_STRING   "string,"
#define VAR_STRING      "var,"
\end{verbatim}}

An attribute may be either a string or a numeric type.  Attributes in a
prototype list are always strings. These attributes are the {\bf
cfinfo} entries for the given device.  Attributes may have one or more
possible values representing user's choices.  The first value of an
attribute is the default value.  An attribute of type {\bf var}
specifies a configurable attribute.  An attribute of type {\bf var} has
for its value the name of the configurable attribute.  When the
application finds an attribute of type {\bf var}, the application must
perform the variable substitution and get the attribute data for the
attribute specified by the {\bf var} attribute.  See {\bf
device.cfinfo(4)} for details.  The application retrieves the values
for an attribute with the following routines. \\

{\small \tt \noindent{\begin{tabular}{@{}ll@{}}
int & count\_numeric(char* fmt); \\
int & count\_string(char* fmt); \\
void &next\_numeric(char** var\_fmt, int* pn); \\
char* & next\_string(char** var\_fmt); \\
char* & expand\_abbr(char* text);
\end{tabular}}} \\

Count\_numeric() return the number of choices for a numeric attribute.
Count\_string() returns the number of choices for a string attribute.
Next\_numeric() stores the next numeric value of a numeric attribute in
its argument.  Next\_string() returns the next string value for a
string attribute.  All attribute names must be expanded via
expand\_abbr().  For example, the application may process all numerical
choices in a prototype string as follows:

{\small \begin{verbatim}
        val_list_t* var_typ,
        char*       typ_str;
        int         nchoices;
        int         n;
        int         i;

        typ_str = var_typ->val.string;
        nchoices = count_numeric(var_typ->val.string);

        for (i = 0; i < nchoices; i++) {
                next_numeric(&typ_str, &n);
                /* process the value */
        }
\end{verbatim}}

\section{Configuration update}

{\small \tt \noindent{\begin{tabular}{@{}ll@{}}
int & modified\_conf(void); \\
void & update\_conf(void);
\end{tabular}}} \\

The application must sync the final device configuration data system
files.  Modified\_conf() checks if the device configuration data has
been modified before doing the update.  Update\_conf() does the  actual
update.

\section{User interface}

The application must provide the following routines for the user
interface portion of the DVC API. \\

{\small \tt \noindent{\begin{tabular}{@{}ll@{}}
void & ui\_notice(char *text); \\
void & ui\_error\_exit(char *text);
\end{tabular}}} \\

Ui\_notice() gives the user a non-fatal warning message.
Ui\_error\_exit() gives the user a fatal error and exits.

\end{document}}
